% Style.
\documentclass[letterpaper,portuguese,12pt,pdftex]{exam}

\usepackage{setspace}
\usepackage{lineno}
\usepackage[left=2.5cm,top=3cm,right=2.5cm]{geometry}

% Portuguese.
\usepackage[brazil]{babel}
\usepackage[T1]{fontenc}
\usepackage[utf8x]{inputenc}
\usepackage{textcomp}

% Font.
\usepackage{lmodern}

% Figures.
\usepackage{epsf,epsfig}

% Bibtex and extras.
\usepackage{natbib}
\usepackage{url}
\usepackage[bookmarks=false,colorlinks=true,urlcolor={green},linkcolor={green},pdfstartview={XYZ null null 1.22}]{hyperref}

% Math.
\usepackage{amssymb}
\usepackage{amsmath}
\usepackage{mathtools}
\everymath{\displaystyle}

% Exam.
\addpoints
\printanswers
% \noprintanswers
\usepackage{color}
\definecolor{SolutionColor}{rgb}{0.8,0.9,1}
\shadedsolutions
\renewcommand{\solutiontitle}{\noindent\textbf{Solução:}\par\noindent}
\pagestyle{headandfoot}
\footer{}{Página \thepage\ de \numpages}{}
\boxedpoints
\pointsinrightmargin
\pointpoints{ponto}{pontos}
\hqword{Questão}
\hpword{Pontos}
\hsword{Nota}
% \qformat{\textbf{Question\thequestion}\quad(\thepoints)\hfill}

% User commands.
\newcommand{\pd}[2]{\dfrac{\partial #1}{\partial #2}}

% PDF metadata.
\pdfinfo{% hyperref overrides this
  /Title    (Prova 01 -- Ondas e Marés)
  /Author   (Filipe Fernandes)
  /Creator  (Filipe Fernandes)
  /Producer (Filipe Fernandes)
  /Subject  (prova)
  /Keywords (prova, oceanografia)
}

% Front page.
\title{Prova 01 -- Ondas e Marés}
\author{Prof. Filipe Fernandes}
\date{04-Oct-2013}

\begin{document}
\maketitle
\doublespacing

\vspace{1cm}
\hbox to \textwidth{Nome e número de matrícula:\enspace\hrulefill}
\vspace{1cm}

\begin{minipage}{.8\textwidth}
Esse exame incluí \numquestions\ questões. O número total de pontos é \numpoints. % 25
\end{minipage}


\begin{questions}

\question
Identifique a {\bf amplitude}, o {\bf comprimento}, a {\bf frequência},
o {\bf período} e o {\bf número de onda} nas formas abaixo.

\begin{parts}
  \part[3]
  $ \eta = 10 \cos(4x + 2t) $

  \begin{solution}
    a = 10, k = 4, L = $\frac{2\pi}{k} = \frac{\pi}{2}$, $\omega$ = −2,
    T = $\frac{2\pi}{\omega} = −\pi$
  \end{solution}

  \part[3]
  $ \eta = −0.5 \sin(\pi x − t) $

  (Ponto extra para quem re-escrever a segunda onda utilizando a função cosseno.)

  \begin{solution}
  a = −0.5, k = $\pi$, L = $\frac{2\pi}{k}$ = 2, $\omega$ = 1,
  T = $\frac{2\pi}{\omega} = 2\pi$

  {\bf Ponto extra:}
  Essa onda pode ser re-escrita na forma,  $\eta = −0.5 \cos(\pi x − t + \pi/2)$
  \end{solution}
\end{parts}


\question
Dada a relação de dispersão de ondas de gravidade,

\[
  \omega^2 = gk\tanh(kh),
\]


e as seguintes informações sobre a tangente hiperbólica e a relação entre
profundidade $h$ e comprimento de onda $L$:

{\bf Ondas longas (ou ondas de águas rasa):}
\begin{itemize}
  \item $\tanh(kh) \sim kh; h << L$
  \item $\frac{h}{L} < \frac{1}{20}$
\end{itemize}

{\bf Ondas curtas (ou ondas de água profunda):}
\begin{itemize}
  \item $\tanh(kh) \sim 1; h >> L$
  \item $\frac{h}{L} > \frac{1}{2}$
\end{itemize}

Responda:

\begin{parts}
  \part[4] Faça a aproximação para águas rasa (ondas longas) e águas profundas
  (ondas curtas) na {\bf relação de dispersão} e crie uma tabela com as equações
  para $C$ e $Cg$ em cada um dos regimes de ondas.

  \begin{solution}
    \raggedright
    {\bf Onda longa (água rasa):}\\
    Dispersão: $\omega = k\sqrt{gh}$\\
    Vel. de Fase: $C = \dfrac{\lambda}{T} = \dfrac{\omega}{k} = \sqrt{gh}$\\
    Vel. de Grupo: $Cg = \pd{}{k}\left(\omega = k\sqrt{gH}\right)
    \rightarrow C_g = \sqrt{gh}$\\

    {\bf Ondas curtas (água profunda):}\\
    Dispersão: $\omega = \sqrt{gk}$\\
    Vel. de Fase: $C = \dfrac{\lambda}{T} = \dfrac{\omega}{k} = \sqrt{g/k}$\\
    Vel. de Grupo: $C_g = \pd{}{k}\left[\omega = g^{1/2}k^{1/2}\right]
    \rightarrow C_g = \dfrac{1}{2}g^{1/2}k^{-1/2} = \dfrac{C}{2}$\\
  \end{solution}

  \part[2]
  Explique o fenômeno de {\bf dispersão} e {\bf refração} de ondas usando a
  {\bf velocidade de fase} ($C$), a {\bf velocidade de grupo} ($C_g$) e a
  {\bf profundidade} (h).

  \begin{solution}
    {\bf Dispersão:} Ocorre devido a velocidade de fase ($C$) de ondas curtas
    ser o dobro da velocidade de grupo ($C_g$), assim algumas ondas ``escapam''
    do grupo levando a seleção das ondas.

    {\bf Refração:} A refração ocorre sempre que uma parte da onda longa estiver
    em profundidades diferentes ($h$), assim partes diferentes da mesma onda
    terão velocidades de fase (e de grupo $C=C_g=\sqrt{gh}$) diferente forçando
    a mesma a girar.
  \end{solution}

  \part[2]
  Para chegar na relação de dispersão e na solução de águas profundas fizemos
  várias {\bf aproximações}.  Cite {\bf 3} aproximações e explique sua validade
  {\bf física}, onde/porque ela é {\bf aceitável} e quando (se algum momento)
  ela pode ser {\bf invalidada}.

  \begin{solution}
    \begin{itemize}
      \item Longe o sítio da forçante do vento
      \item Longe da zona de arrebentação e quebra;
      \item Dimensões horizontais infinitas;
      \item Sem atrito;
      \item Período da onda muito menor que o período inercial;
      \item Assume-se um estado médio e perturbações sobre esse (as ondas);
      \item A amplitude da onda é pequena quando comparada com a coluna d'água.
    \end{itemize}
  \end{solution}

  \part[2]
  Quando assumimos uma forma de onda para a solução da
  {\bf amplitude de pressão} $\mathbb{P}$,

  \begin{equation}
    \mathbb{P}(z) = \cos(\theta),
    \label{eq:P}
  \end{equation}

  na equação de Laplace,

  \begin{equation}
    \nabla^2 \tilde{p} = 0,
    \label{eq:Laplace}
  \end{equation}

  resultamos na Equação Diferencial de Segunda Ordem Homogênea abaixo:

  \begin{equation}
    \frac{\partial^2\mathbb{P}}{\partial z^2} - \mathbf{K}^2\mathbb{P}=0
    \label{eq:dif}
  \end{equation}

  Para resolvermos essa equação precisamos de duas condições de contorno.
  Explique:

  \begin{itemize}
    \item Por que {\bf não} utilizamos condições de contorno em todas as
          dimensões $x$, $y$, $z$? Onde colocamos as condições de contorno?
          {\bf Explique a sua resposta!}
    \item Ambas condições de contorno são {\bf dinâmicas}, {\bf ou estáticas}?
          Se for(em) dinâmica(s), como fazemos para ``{\bf acompanhar}'' a
          variável enquanto ela muda?
  \end{itemize}

  (Ponto extra: Substitua a equação \ref{eq:P} em \ref{eq:Laplace} e chegue na
  diferencial \ref{eq:dif} lembrando que $\mathbf{K} = \sqrt{k^2 + l^2}$.)

  \begin{solution}
    1) Utilizamos condições de contorno apenas na {\bf dimensões vertical}
    ($z$), deixando as dimensões horizontais ($x$, $y$) como ``infinitas''.
    Isso porque as ondas se manifestam como {\bf alterações verticais na
    pressão.}

    2) {\bf Não}.  Na superfície usamos a condição {\bf dinâmica} com
    continuidade de pressões.  Já fundo usamos condição {\bf estática} sem
    movimento ``cruzando'' o fundo).

    3) Na condições de contorno dinâmica usamos uma expansão em {\bf Série de
    Taylor} para centrar a condição de contorno em $z=0$.
  \end{solution}
\end{parts}

\question
Cálculos básicos de ondas.

\begin{parts}
  \part[2]
  Se {\bf 16 cristas} de ondas passam {\bf sucessivamente} por um ponto fixo num
  intervalo de {\bf 1 minuto e 40 segundos}, qual é a {\bf frequência angular}
  dessas ondas?

  \begin{solution}
    $T = \dfrac{60 + 40}{16} = 6.25$ s

    $\omega = \dfrac{2\pi}{6.25} \sim 1$ s$^{-1}$
  \end{solution}

  \part[2]
  O {\bf período} de uma onda é de {\bf 25 segundos}.  Qual seria a
  {\bf velocidade} dessa onda em {\bf águas profundas}?

  \begin{solution}
    {\bf Não há} relação entre $k$ e $T$, logo, sem a informação do comprimento
    de onda $L$ ou do número de onda não temos como calcular!
  \end{solution}

  \part[2]
  Qual seria a {\bf velocidade} de uma onda com {\bf comprimento} de ondas
  de 312 m em {\bf águas profundas}?  E em {\bf águas rasas}?

  \begin{solution}
    Agora sim podemos calcular para águas profundas (ondas curtas)!

    $C = \sqrt{g/k} = \sqrt{9.8 / 0.02013} = 22.06$ m s$^{-1}$

    Já para águas rasas (ondas longas) temos como calcular o limite superior
    das velocidades.

    $\dfrac{h}{312} < \dfrac{1}{20} \therefore h <  15.6$ m

    $C < \sqrt{9.8 \times 15.6}$

    $C < 152.88$ m s$^{-1}$
  \end{solution}
\end{parts}

\question[3]
Sabemos que a geração de ondas de superficiais de gravidade está associada ao
vento.  Cite os {\bf 3 fatores} que precisamos saber sobre o {\bf vento} para
estimar a quantidade de energia que será transferida para as ondas geradas.

(Ponto extra: Há algum fator {\bf limitante} que não está associado ao vento?)

\begin{solution}
  \begin{itemize}
    \item A velocidade do vento;
    \item Pista do vento (a distância em que o vento sopra);
    \item Duração do vento;
  \end{itemize}
  {\bf Ponto extra:} Profundidade da coluna d'água é um fator limitante
  (ressacas costeiras).
\end{solution}

\end{questions}

\end{document}
