% Style.
\documentclass[letterpaper,portuguese,12pt,pdftex]{exam}

\usepackage{setspace}
\usepackage{lineno}
\usepackage[left=2.5cm,top=3cm,right=2.5cm]{geometry}

% Portuguese.
\usepackage[brazil]{babel}
\usepackage[T1]{fontenc}
\usepackage[utf8x]{inputenc}
\usepackage{textcomp}

% Font.
\usepackage{lmodern}

% Figures.
\usepackage{epsf,epsfig}

% Bibtex and extras.
\usepackage{natbib}
\usepackage{url}
\usepackage[bookmarks=false,colorlinks=true,urlcolor={green},linkcolor={green},pdfstartview={XYZ null null 1.22}]{hyperref}

% Math.
\usepackage{amssymb}
\usepackage{amsmath}
\usepackage{mathtools}
\everymath{\displaystyle}

% Exam.
\addpoints
% \printanswers
\noprintanswers
\usepackage{color}
\definecolor{SolutionColor}{rgb}{0.8,0.9,1}
\shadedsolutions
\renewcommand{\solutiontitle}{\noindent\textbf{Solução:}\par\noindent}
\pagestyle{headandfoot}
\footer{}{Página \thepage\ de \numpages}{}
\boxedpoints
\pointsinrightmargin
\pointpoints{ponto}{pontos}
\hqword{Questão}
\hpword{Pontos}
\hsword{Nota}
% \qformat{\textbf{Question\thequestion}\quad(\thepoints)\hfill}

% User commands.
\newcommand{\pd}[2]{\dfrac{\partial #1}{\partial #2}}

% PDF metadata.
\pdfinfo{% hyperref overrides this
  /Title    (Prova 01 -- Ondas e Marés)
  /Author   (Filipe Fernandes)
  /Creator  (Filipe Fernandes)
  /Producer (Filipe Fernandes)
  /Subject  (prova)
  /Keywords (prova, oceanografia)
}

% Front page.
\title{Prova 01 -- Ondas e Marés}
\author{Prof. Filipe Fernandes}
\date{04-Oct-2013}

\begin{document}
\maketitle
\doublespacing

\vspace{1cm}
\hbox to \textwidth{Nome e número de matrícula:\enspace\hrulefill}
\vspace{1cm}

\begin{minipage}{.8\textwidth}
Esse exame incluí \numquestions\ questões. O número total de pontos é \numpoints.
\end{minipage}

% 25
\begin{questions}
\question
Identifique a amplitude, número de onda, comprimento de onda, frequência de onda,
e período de onda nas formas abaixo.  Dica: Lembrem-se que $x$ representa o
nosso eixo de ``espaço'' e $t$ o eixo do ``tempo''.

 \begin{parts}
  \part[3]
  $\eta = 10 \cos(4x + 2t)$

  \begin{solution}
    a = 10, k = 4, L = $\frac{2\pi}{k} = \frac{\pi}{2}$, $\omega$ = −2,
    T = $\frac{2\pi}{\omega} = −\pi$
  \end{solution}

  \part[3]
  $\eta = −0.5 \sin(\pi x − t)$

  Ponto extra para quem re-escrever essa onda utilizando $\cos$ + uma fase
  $\phi$.

  \begin{solution}
    a = −0.5, k = $\pi$, L = $\frac{2\pi}{k}$ = 2, $\omega$ = 1,
    T = $\frac{2\pi}{\omega} = 2\pi$

    Essa onda pode ser re-escrita na forma:
    $\eta = −0.5 \cos(\pi x − t + \pi/2)$
  \end{solution}
 \end{parts}


\question
Data a relação de dispersão de ondas de gravidade,

$$\omega^2 = gk\tanh(kh)$$

e as seguintes informações,

ondas longas (ou ondas de águas rasa):
\begin{itemize}
  \item $\tanh(kh) \sim kh; h << L$
  \item $\frac{h}{L} < \frac{1}{20}$
\end{itemize}

ondas curtas (ou ondas de água profunda):
\begin{itemize}
  \item $\tanh(kh) \sim 1; h >> L$
  \item $\frac{h}{L} > \frac{1}{2}$
\end{itemize}

Responda:

\begin{parts}
  \part[4] Faça a aproximação para águas rasa (ondas longas) e águas profundas
  (ondas curtas) e crie uma tabela com as equações para $\omega$, L, $C$ e $Cg$
  para cada um dos regimes de ondas.

\begin{solution}
\raggedright
      Onda longa (água rasa)\\
      Dispersão: $\omega^2 = gk^2h$\\
      Vel. de Fase: $C = \dfrac{\lambda}{T} = \dfrac{\omega}{k} = \sqrt{gh}$\\
      Vel. de Grupo: $Cg = \pd{}{K}\left(\omega = K\sqrt{gH}\right) \rightarrow C_g = \sqrt{gh}$\\

      Ondas curtas (água profunda)\\
      Dispersão: $\omega^2 = gk$\\
      Vel. de Fase: $C = \dfrac{\lambda}{T} = \dfrac{\omega}{k} = \sqrt{g/k}$\\
      Vel. de Grupo: $C_g = \pd{}{k}\left[\omega = g^{1/2}k^{1/2}\right] \rightarrow C_g = \dfrac{1}{2}g^{1/2}k^{-1/2} = \dfrac{C}{2}$\\
  \end{solution}

  \part[2]
  Explique o fenômeno de dispersão e refração usando a relação entre $C$ e
  $C_g$.

  \begin{solution}
  TODO
  \end{solution}

  \part[2]
  Para chegar na relação de dispersão e na solução de águas profundas fizemos
  várias aproximações.  Cite 3 e explique seu princípio, sua validade física,
  onde ela é aceitável e quando (se algum momento) ela pode ser invalidada.

  \begin{solution}
    \begin{itemize}
      \item Longe o sítio da forçante do vento.
      \item Sem atrito.
      \item Período da onda muito menor que o período inercial.
      \item Assume-se um estado médio e perturbações sobre esse (as ondas).
      \item A amplitude da onda é pequena quando comparada com a coluna d'água.
    \end{itemize}
  \end{solution}

  \part[2]
  Quando assumimos uma forma de onda para a solução da amplitude de pressão,

  \begin{equation}
    \mathbb{P}(z) = \cos(\theta),
    \label{eq:P}
  \end{equation}

  para a equação de a equação de Laplace,

  \begin{equation}
    \nabla^2 \tilde{p} = 0,
    \label{eq:Laplace}
  \end{equation}

  resultamos em uma Equação Diferencial de Segunda Ordem Homogênea para a
  amplitude de pressão abaixo:

  \begin{equation}
    \frac{\partial^2\mathbb{P}}{\partial z^2} - \mathbf{K}^2\mathbb{P}=0
    \label{eq:dif}
  \end{equation}

  Para resolvermos essa equação precisamos de duas condições de contorno no
  mínimo.  Explique:
  \begin{itemize}
    \item Onde colocamos as condições de contorno?
    \item Ambas são dinâmicas, ou estáticas?  Se for(em) dinâmica(s), como
    fazemos para ``acompanhar'' a variável enquanto ela muda?
  \end{itemize}

  (Ponto extra, substitua a equação \ref{eq:P} em \ref{eq:Laplace} e chegue na
  diferencial \ref{eq:dif} lembrando que $\mathbf{K} = \sqrt{k^2 + l^2}$.)


  \begin{solution}
  1) Superfície (dinâmica, continuidade de pressões) e fundo (estática, sem
     movimento ``cruzando'' o fundo).
  2) Série de Taylor.
  \end{solution}
\end{parts}

  \question
  Cálculos básicos de ondas.
 \begin{parts}
  \part[2]
  Se 16 cristas de ondas passam sucessivamente por um ponto fixo num
  intervalo de 1 minuto e 40 segundos, qual é a frequência angular dessas ondas?

  \begin{solution}
  $\omega = \dfrac{16}{60 + 40} = 0.16$ s$^{-1}$
  \end{solution}

  \part[2]
  O período de uma onda é de 25 segundos.  Qual seria a velocidade dessa onda
  em águas profundas?

  \begin{solution}
  TODO
  \end{solution}

  \part[2]
  Qual seria a velocidade de uma onda com comprimento de ondas de 312 m em águas
  profundas?  E em águas rasas?

  \begin{solution}
  TODO
  \end{solution}

\end{parts}

\question[3]
% \begin{parts}
%   \part[3]
  Sabemos que a geração de ondas de superficiais de gravidade está associada ao
  vento.  Cite os {\bf 3 fatores} que precisamos saber sobre o {\bf vento} para
  estimar a quantidade de energia que será transferida para as ondas geradas.

  (Ponto extra: Há algum fator {\bf limitante} que não está associado ao vento?)

  \begin{solution}
  A velocidade do vento;
  Pista do vento ou a distância em que o vento sopra;
  Duração do vento;
  Profundidade da água.
  \end{solution}

%  \end{parts}

\end{questions}

\end{document}
