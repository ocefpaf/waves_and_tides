\documentclass[11pt,portuguese,a4paper,pdftex]{article}

\usepackage[brazil]{babel}
\usepackage{amssymb,amsmath}
\usepackage[utf8x]{inputenc}
\usepackage{nicefrac}
\usepackage{fancyhdr}
\usepackage{multirow}
\usepackage{color}
\usepackage{helvet}

\renewcommand{\baselinestretch}{1.5}
\topmargin -6mm
\hoffset -11mm
\textwidth 160mm
\textheight 240mm

\renewcommand{\headrulewidth}{0.0pt}
\renewcommand{\footrulewidth}{0.0pt}

\begin{document}

{\Large
\begin{center}
  Ondas e Marés\\
  \vspace{0.3 cm}
  Prof. Filipe Fernandes\\
  25 de Outubro de 2013\\
\end{center}
}

\begin{center}
  {\Large Aula Prática}
\end{center}

 \begin{center}
{\bf \Large Protocolo para Análise Harmônica de Marés}\\
  \end{center}

\vspace{0.5cm}

\hspace{-8mm} Pré-processamento dos dados:

\begin{enumerate}
  \item Controle de qualidade dos dados coletados, reconhecimento e remoção de
        dados incorretos.

  \item Remoção da média para (Análise Harmônica é feita sobre as anomalias da
        elevação).

  \item Análise Harmônica para identificar as principais componentes de maré,
        suas amplitudes e fases.
\end{enumerate}

\hspace{-8mm} Análise dos dados:

\begin{enumerate}
  \item Calcular o número de forma e classificar a maré quanto ao seu tipo.
  \begin{eqnarray}\label{eq:forma}
    F = \frac{K_1 + O_1}{M_2 + S_2}
  \end{eqnarray}

  \begin{center}
    \begin{tabular}{|c|l|}\hline
      $0 < F < 0.25$ & maré semi-diurna \\
      $0.25 < F < 1.5$ & mista, com maioria semi-diurna \\
      $1.5 < F < 3$ & mista, com maioria diurna \\
      $F > 3$ & maré diurna\\ \hline
    \end{tabular}
  \end{center}

  \item Calcula as diferenças de fase entre algumas componentes e compará-las.

  \item Estimar uma partição de energia entre elevação observada devido à maré e
        a não devido a maré.  (Utilizaremos um filtro passa-baixa de 40 h.)

%   \item Utilizar as equações (\ref{eq:eta})-(\ref{eq:uv}) para estimar valores
%         de elevação da superfície do mar devida a maré ($\eta$) e das correntes
%         de maré (componentes $u$ e $v$).
%
%         \begin{eqnarray}\label{eq:eta}
%           \frac{\eta(x)}{\eta(0)} = 1 -
%           \left(\frac{w^2 - f_o^2}{g\ \beta_1} + \frac{f_o\ l_r}{w}  \right).x\ ;
%           \ \ \ \ \ \ l_r = \left(\frac{\Delta G}{\Delta y}   \right)_{x=0}\ ,
%         \end{eqnarray}
%
%   onde:\\
%   $\eta(0):$ nível do mar na costa;\\
%   $\eta(x):$ nível do mar em um ponto qualquer ao largo da PC;\\
%   $\beta_1:$ inclinação do fundo (tangente do ângulo de inclinação);\\
%   $l_r:$ razão entre as diferenças de fase de duas estações costeiras;\\
%   $G:$ fase em relação a Greenwich (obtida na análise harmônica);
%   $y:$ distância ao longo da costa.
%
%   \begin{eqnarray}\label{eq:uv}
%     u = \frac{w}{\beta_1} \eta(0)\ \sin(ly-wt)\ ;
%     \ \ \ \ \ \ v = \frac{f_o}{\beta_1} \eta(0)\ \cos(ly-wt)
% \end{eqnarray}


%   \item Utilizar os dados filtrados para calcular correlações entre as elevações
%         de Ubatuba e Cananéia, em diversos {\it lags} de tempo.  Verificar em
%         que {\it lag} ocorre a correlação máxima.\\
\end{enumerate}

% \begin{center}
%   {\bf \Large Investigando o papel do vento}\\
% \end{center}
%
% Conforme visto em classe, o vento paralelo à costa é muito mais importante do
% que o perpendicular no tocante a geração de correntes e elevações junto a costa
% e na plataforma continental.  No cenário real, na maioria das vezes, o vento não
% está orientado especificamente em uma destas direções.  Com isso, é necessário
% decompormos o vetor do vento nas direções perpendicular e paralela à costa.
%
% Dentro do regime típico de ventos vigente na PC sudeste, analisaremos dois
% cenários principais:
%
% {\bf (i) Bom tempo:} Regido pelo centro de alta pressão semi-permanente do
% Atlântico Sul, que proporcionam em média ventos de ENE ao largo de Ubatuba, com
% magnitude entre 4,6-1,5 m s$^{-1}$.
%
% {\bf (ii) Mau tempo:} Vento de S-SW geralmente intenso, logo em seguida à
% passagens de sistemas frontais frios, com magnitude variando entre 7-8 m s$^{-1}$
% 24 h após a passagem e entre 3-7 m s$^{-1}$ 32 h após a passagem, segundo
% {\it Coelho (2008)}.
%
% Utilizem (\ref{eq:V}) para estimar o transporte por unidade de distância
% associado a cada um dos cenários descritos acima, em função do tempo.  Plote os
% resultados e observe o decaimento provocado pelo atrito de fundo.
%
% \begin{eqnarray}\label{eq:V}
%   V = \frac{u_*H_o}{c_d} \left[
%   \frac{1 - exp \left(-2u_*t \frac{\sqrt{c_d}}{H_o} \right)}
%   {1 + exp \left(-2u_*t \frac{\sqrt{c_d}}{H_o} \right)} \right],
% \end{eqnarray}
%
% \vspace{6mm}
%
% Lembrando que $u_*$ é a velocidade friccional, que é definida por
%
% \begin{eqnarray}\label{eq:fric}
%   u_* = \sqrt{\frac{\tau_s}{\bar{\rho}}},\ \ e\ \ que
% \end{eqnarray}
%
% \vspace{-6mm}
%
% \begin{eqnarray}\label{eq:taux}
%   \tau_{sx} = \rho_{ar} c_{10} u_{ar} \sqrt{u^2_{ar} + v^2_{ar}}
% \end{eqnarray}
%
% \vspace{-6mm}
%
% \begin{eqnarray}\label{eq:tauy}
%   \tau_{sy} = \rho_{ar} c_{10} v_{ar} \sqrt{u^2_{ar} + v^2_{ar}}
% \end{eqnarray}
%
% \vspace{6mm}
%
% Em seguida, calcule o $t_a$, que é o tempo de ajuste por atrito, através da Eq.
% (\ref{eq:ta}).  Plote seu valor no gráfico desenhado através de (\ref{eq:V}).
%
% \begin{eqnarray}\label{eq:ta}
%   t_a = \frac{H_o}{2u_* \sqrt{C_d}}
% \end{eqnarray}
%
\end{document}
