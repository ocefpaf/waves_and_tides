% Title page.
\title[Aula 08]{Ondas e marés}
\subtitle{Marés.}
\author[Filipe Fernandes]{Filipe P. A. Fernandes}
\institute[unimonte]{Centro Universitário Monte Serrat}
\date[Novembro 2013]{01 de Novembro 2013}

\logo{\includegraphics[scale=0.15]{../common/university_logo.png}}

\begin{document}

% The title page frame.
\begin{frame}[plain]
  \titlepage
\end{frame}

\section*{Outline}
\begin{frame}
\tableofcontents
\end{frame}

\section{Marés}
\subsection{Descrição}

\begin{frame}

  \frametitle{Antes de começar...}
  \begin{columns}
    \begin{column}{0.5\textwidth}
  \small{
  Frequência (ou Força/Parâmetro/Aceleração) de Coriolis
  \begin{block}{}
    \begin{align}
      f &= 2\Omega\sin(\theta) \\
      f &= f_o 2\Omega\sin(\theta_o) \\
      f &= f_o + \beta y \\
      \beta &= \frac{2\Omega\cos(\theta_o)}{R}
    \end{align}
  \end{block}
  }
    \end{column}
    \begin{column}{0.5\textwidth}
      \begin{center}
        \includegraphics[scale=0.4]{../figures/Coriolis.png}
      \end{center}
    \end{column}
  \end{columns}
\end{frame}

\begin{frame}
  \frametitle{Marés -- Definição}
  \begin{center}
  \begin{block}{}
      É a oscilação vertical da superfície do mar ou outra massa d'água
      na superfície da Terra causada principalmente devido às atrações
      gravitacionais da Lua e do Sol.
  \end{block}
  \end{center}
\end{frame}

\begin{frame}
  \frametitle{Marés -- Definição}
    \begin{itemize}[<+-| alert@+>]
      \item Oscilações periódicas;
      \item Altura máxima da água é denominada preamar (PM);
      \item Menor altura da água é denominada de baixa-mar (BM);
      \item A maré, por ser uma onda de grande comprimento, se comporta como
            uma ondas se propagando em águas rasas.
    \end{itemize}
\end{frame}

\begin{frame}
  \frametitle{Forças geradoras}
    \begin{itemize}[<+-| alert@+>]
      \item Para o sistema Terra-Lua:
      \item \[F = G\frac{M_TM_L}{R^2}\]
      \item Onde,
      \item $G = 6.67 10^{-11}$ N kg${-2}$m$^{2}$
    \end{itemize}
\end{frame}


\begin{frame}
  \frametitle{Forças geradoras -- Força gravitacional}
  \begin{center}
    \includegraphics[scale=0.7]{../figures/gravitational_force.png}
  \end{center}
\end{frame}


\begin{frame}
  \frametitle{Forças geradoras -- Força centrífuga}
  \begin{center}
    \includegraphics[scale=0.6]{../figures/centrifugal_force.png}
  \end{center}
\end{frame}

\subsection{A teoria do equilíbrio das marés}
\begin{frame}
  \frametitle{Forças geradoras -- Maré de equilíbrio}
  \begin{center}
    \includegraphics[scale=0.5]{../figures/gravitation_centrifuge_forces.png}
  \end{center}
\end{frame}

\begin{frame}
  \frametitle{Maré de equilíbrio}
  \begin{block}{}
  Maré de Equilíbrio é a elevação da superfície do mar que estaria em
  equilíbrio com as forças de maré caso a Terra fosse totalmente coberta por
  água e esta água respondesse de forma instantânea à atração gravitacional.
  \end{block}
\end{frame}


\begin{frame}
  \frametitle{Maré de equilíbrio}
  \small{
  \begin{block}{}
  Os resultados obtidos com a Teoria de Equilíbrio são bastante diferentes das
  medições que obtemos na prática. Mesmo assim esta teoria nos permite
  compreender alguns fenômenos que observamos na natureza. Podemos
  determinar, por exemplo, que os períodos das oscilações do nível do mar
  apresentam períodos idênticos às forças atuantes e que suas amplitudes são,
  também, proporcionais à estas forças. Com base nessas considerações
  procuraremos determinar essas periodicidades e precisaremos então
  representar a elevação do nível do mar em termos da declinação da Lua, da
  latitude do lugar, etc.
  \end{block}
  }
\end{frame}

\begin{frame}
  \frametitle{Maré de equilíbrio}
    \includegraphics[scale=0.35]{../figures/equilibrium.png}
\end{frame}

\begin{frame}
  \frametitle{Forças geradoras -- Centro de massa}
  \begin{center}
    \includegraphics[scale=0.7]{../figures/earth_moon.png}
  \end{center}
\end{frame}



\begin{frame}
  \frametitle{Forças geradoras -- Centro de massa}
  \begin{center}
    \includegraphics[scale=0.55]{../figures/moon_delay.png}
  \end{center}
\end{frame}



\begin{frame}
  \frametitle{Forças geradoras -- Centro de massa}
  \begin{center}
    \includegraphics[scale=0.6]{../figures/earth_moon_02.png}
  \end{center}
\end{frame}

\begin{frame}
  \frametitle{Sizígia e quadratura}
  \begin{center}
    \includegraphics[scale=0.45]{../figures/spring_neap.png}
  \end{center}
\end{frame}


\begin{frame}
  \frametitle{Sizígia e quadratura}
  \begin{center}
    \includegraphics[scale=0.45]{../figures/spring_neap_tides.png}
  \end{center}
\end{frame}


\subsection{Nível de redução}
\begin{frame}
  \frametitle{Nível de redução -- NR}
  \begin{block}{}
  Nível a que são referidas as alturas das águas e as sondagens representadas
  nas Cartas Náuticas.  Como o NR adotado pela DHN é normalmente o nível médio
  das baixa-mares de sizígia, geralmente se encontram maiores profundidades que
  as sondagens lançadas na carta; entretanto, por ocasião das BM de sizígia,
  podem ser encontradas profundidades menores que as constantes da carta.
  \end{block}
\end{frame}


\begin{frame}
  \frametitle{Marégrafos}
  \begin{center}
    \includegraphics[scale=0.35]{../figures/maregraph.png}
  \end{center}
\end{frame}

% 9.5 A teoria dinâmica das marés
\section{Classificação das marés}
\begin{frame}
  \frametitle{Principais constituintes de maré}
  \footnotesize{
\begin{table}
    \begin{tabular}{|l|l|}
        \hline
        $M_2$ & 12 h 25 min (1/2 dia lunar) -- Maré semidiurna produzida pela Lua \\ \hline
        $S_2$ & 12 h (1/2 dia solar) -- Maré semidiurna produzida pelo Sol        \\
        $N_2$ & 12.66 horas -- Maré induzida pela elíptica lunar                  \\
        $O_1$ & 24 h 50 min -- Maré diurna produzida pela Lua                     \\
        $K_1$ & 24 -- Maré diurna produzida pelo Sol                              \\
        \hline
    \end{tabular}
\end{table}
}
\end{frame}

\begin{frame}
  \frametitle{Classificação das marés}
    \begin{columns}
    \column{4cm}
    \begin{block}{}
      \[ F = \frac{K_1 + O_1}{M_2 + S_2} \]
    \end{block}
    \end{columns}
\scriptsize{
\begin{table}
    \begin{tabular}{|l|p{9cm}|}
        \hline
        $F < 0.25$       & \parbox{9cm}{Marés Semidiurnas: duas altas e duas baixas com aproximadamente a mesma altura.  A média da maré de sizígia = $2(M_2 + S_2)$} \\ \hline
        $0.25 > F > 1.5$ & \parbox{9cm}{Marés mista com predominância semidiurna: grandes inequalidades na amplitude da maré e intervalo entre altas e baixas.  A média da maré de sizígia = $2(M_2 + S_2)$} \\ \hline
        $1.5 > F > 3.0$  & \parbox{9cm}{Marés mista com predominância diurna: Em geral apenas uma maré alta por dia.  A média da maré de sizígia = $2(K_1 + O_1)$}                                           \\ \hline
        $F > 3.0$        & \parbox{9cm}{Marés diurna: Geralmente apenas uma maré alta por dia. A média da maré de sizígia = $2(K_1 + O_1)$}                                                                 \\ \hline
    \end{tabular}
\end{table}
}
\end{frame}


\begin{frame}
  \frametitle{Número de forma}
  \begin{center}
    \includegraphics[scale=0.5]{../figures/form_factor.png}
  \end{center}
\end{frame}


\begin{frame}
  \frametitle{Exemplo}
  \begin{center}
    \includegraphics[scale=0.5]{../figures/tidal_components.png}
  \end{center}
\end{frame}


\begin{frame}
  \frametitle{Tipos de Maré}
  \begin{center}
    \includegraphics[scale=0.4]{../figures/types_of_tides.png}
  \end{center}
\end{frame}


\begin{frame}
\frametitle{Dever de casa}
%     TODO
\end{frame}

\end{document}
