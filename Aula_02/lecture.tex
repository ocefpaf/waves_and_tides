% \documentclass[14pt,xcolor=dvipsnames]{beamer}
\documentclass[handout,14pt,xcolor=dvipsnames]{beamer}

% Notes.
\usepackage{handoutWithNotes}  % Handout mode.
\pgfpagesuselayout{4 on 1 with notes}[a4paper,border shrink=5mm]
% \pgfpagesuselayout{1 on 1 with notes landscape}[a4paper,border shrink=5mm]

\usepackage{lmodern}

% See http://goo.gl/p0Phn for other colors
\usecolortheme[named=OliveGreen]{structure}  % Specify base color
% \setbeamercolor{structure}{fg=OliveGreen!50!black}

% See deic.uab.es/~iblanes/beamer_gallery/index_by_theme.html for other themes.
\usetheme{Madrid}
% \usetheme{Warsaw}

% Specify other colors and options as required.
\setbeamercolor{alerted text}{fg=Maroon}
\setbeamertemplate{items}[square]
% \setbeamertemplate{navigation symbols}{}  % Remove lower navigation panel.
\useoutertheme{infolines}  % Add info lines at the bottom.

% Packages.
\usepackage{amssymb,amsmath}
\usepackage{fancybox}  % Box [shadowbox, fbox, doublebox, ovalbox, Ovalbox].
\usepackage[framed,numbered]{mcode}
\usepackage{multimedia}
\setbeamertemplate{blocks}[rounded][shadow=true]
\usepackage{textcomp}
\usepackage[utf8]{inputenc}
\usepackage{graphicx}

% Progress bar.
\usepackage{tikz}
\usetikzlibrary{calc}
\definecolor{pbgray}{HTML}{575757}  % Background color for the progress bar.

\makeatletter
\def\progressbar@progressbar{}  % Progress bar.
\newcount\progressbar@tmpcounta  % Auxiliary counter.
\newcount\progressbar@tmpcountb  % Auxiliary counter.
\newdimen\progressbar@pbht  % Progressbar height.
\newdimen\progressbar@pbwd  % Progressbar width.
\newdimen\progressbar@tmpdim  % Auxiliary dimension.

\progressbar@pbwd=\linewidth
\progressbar@pbht=1pt

% Progress bar.
\def\progressbar@progressbar{%

    \progressbar@tmpcounta=\insertframenumber
    \progressbar@tmpcountb=\inserttotalframenumber
    \progressbar@tmpdim=\progressbar@pbwd
    \multiply\progressbar@tmpdim by \progressbar@tmpcounta
    \divide\progressbar@tmpdim by \progressbar@tmpcountb

  \begin{tikzpicture}[very thin]
    \draw[pbgray!30,line width=\progressbar@pbht]
      (0pt, 0pt) -- ++ (\progressbar@pbwd,0pt);
    \draw[draw=none]  (\progressbar@pbwd,0pt) -- ++ (2pt,0pt);

    \draw[fill=pbgray!30,draw=pbgray] %
       ( $ (\progressbar@tmpdim, \progressbar@pbht) + (0,1.5pt) $ ) -- ++(60:3pt) -- ++(180:3pt) ;

    \node[draw=pbgray!30,text width=3.5em,align=center,inner sep=1pt,
      text=pbgray!70,anchor=east] at (0,0) {\insertframenumber/\inserttotalframenumber};
  \end{tikzpicture}
}

\addtobeamertemplate{headline}{}
{
  \begin{beamercolorbox}[wd=\paperwidth,ht=5ex,center,dp=1ex]{white}%
    \progressbar@progressbar
  \end{beamercolorbox}
}
\makeatother

% Title page.
\title[Aula 02]{Ondas e Marés}
\subtitle{Introdução}
\author[Filipe Fernandes]{Filipe P. A. Fernandes}
\institute[unimonte]{Centro Universitário Monte Serrat}
\date[Agosto 2012]{20 de Agosto 2012}

\logo{\includegraphics[scale=0.4]{../figures/unimonte.png}}

\begin{document}

% The title page frame.
\begin{frame}[plain]
  \titlepage
\end{frame}

\section*{Outline}
\begin{frame}
\tableofcontents
\end{frame}

\section{Aula 02 -- Introdução}
\section{{Introdução: O que são ondas e como são geradas?}}
\begin{frame}
\frametitle{Introdução: O que são ondas e como são geradas?}
    \begin{block}{}
    Ondas são deformações periódicas em uma interface.  Em oceanografia, ondas
    de superfície são deformações da superfície dos oceanos, i.e., na interface
    oceano-atmosfera.

    As deformações se propagam com a velocidade de onda,
    enquanto as partículas descrevem movimentos oscilatórios ou orbitais com
    velocidade de partículas e permanecem, em média, na mesma posição.
    \end{block}
\end{frame}

\begin{frame}
\frametitle{Introdução: O que são ondas e como são geradas?}
    \begin{block}{}
    Em águas profundas, os caminhos das partículas são circulares. Em águas
    mais rasas, os caminhos percorridos pelas partículas se achatam e ficam
    parecendo elipses.  Uma onda de águas profundas passa a ser considerada
    como de águas rasas quando o comprimento de onda $\lambda$ se torna maior
    do que o dobro da profundidade local da água $h$.  As mudanças nas
    propriedades das ondas entretanto ocorrem antes disso, quando
    $\lambda = 20 h$
    \end{block}
\end{frame}

\begin{frame}
\frametitle{ Teoria linear de ondas de gravidade livres}
\small{
    \begin{block}{}
        Considera o movimento oscilatório (ondulatório) apenas a partir do
        momento em que a forçante que causou o movimento já não existe mais
        ({\bf Sem forçante!}).
    \end{block}
    \pause
    \begin{block}{}
        Além disso, como o período das mesmas é bem menor que o período
        necessário para que o efeito de atenuação causado pelo atrito interno
        do fluido seja considerável. ({\bf Sem atrito!})
    \end{block}
    \pause
    \begin{block}{}
        Ondas de gravidade tem um período muito curto quando comparado ao
        período inercial, e portanto os efeitos de rotação da Terra podem ser
        negligenciados. ({\bf Sem $f$!})
    \end{block}
}
\end{frame}

\begin{frame}
\frametitle{Vai pro quadro preguiçoso!}
\end{frame}

\section{Definições}
\begin{frame}
    \frametitle{Definições}
    \includegraphics[scale=0.45]{../figures/caracterizacao_ondas.png}
\end{frame}

\subsection{Elevação}
\subsection{Altura de onda}
\subsection{comprimento de onda}
\subsection{Número de onda}
\subsection{Período e frequência}
\subsection{Frequência angular}
\begin{frame}
\frametitle{Descrição das Ondas}
    \begin{block}{}
        A maneira mais simples de encarar as ondas é pelo conceito da onda como
        uma oscilação harmônica. Assim, podemos descrevê-las por
    \end{block}
    \begin{itemize}[<+-| alert@+>]
        \item período T
        \item frequência $\omega = 2 \pi / \text{T}$
        \item comprimento de onda $\lambda$
        \item velocidade $c = \lambda / \text{T}$
        \item altura H = 2 A (A = amplitude)
        \item inclinação $\delta = \text{H} / \lambda$
    \end{itemize}
\end{frame}

\begin{frame}
\frametitle{Ondas longas vs ondas curtas}
\small{
\begin{block}{}
$0 \underbrace{< \lambda < 2}_{\substack{\text{ondas de água profunda}\\\text{ou ondas curtas}}} \text{h} < \underbrace{\lambda}_{\text{ondas de transição}} < 20 \text{h} < \underbrace{\lambda}_{\substack{\text{ondas de água rasa}\\\text{ou ondas longas}}}$
\end{block}
}
\end{frame}

\begin{frame}
\frametitle{Ondas de água profundas (ondas curtas)}
    \begin{center}
        \includegraphics[scale=0.35]{../figures/deepwaterwaves.png}
    \end{center}
\end{frame}

\begin{frame}
\frametitle{Ondas longas vs ondas curtas}
    \begin{center}
        \includegraphics[scale=0.45]{../figures/deep_to_shallow.png}
    \end{center}
\end{frame}

\begin{frame}
\frametitle{Ondas de água rasa (ondas longas)}
    \begin{center}
        \includegraphics[scale=0.4]{../figures/shallow_wave.png}
    \end{center}
\end{frame}


\section{Velocidade de fase}
%TODO

%TODO: Ir para o quadro.
\section{Velocidade de grupo}
\begin{frame}
\frametitle{Superposição de ondas}
    \begin{block}{}
    A superposição de duas ondas com frequências bem próximas $\omega_1$ e
    $\omega_2$ respectivamente, produz grupos de ondas ou séries.  As cristas
    individuais propagam com uma velocidade de fase (idêntica a velocidade da
    onda) $c$;  séries de ondas propagam com uma velocidade de grupo:
    \end{block}
    \pause
    \begin{center}
        $c_g = c - \lambda \frac{dc}{d\lambda}$ ou\\
        $c_g = \frac{d\omega}{d\kappa}$
    \end{center}
\end{frame}

\begin{frame}
\frametitle{Chega, tá na hora de vocês trabalharem!}
    \begin{block}{}
    Dever 01: O que são ondas capilares e por quê estuda-las?
    \end{block}
\end{frame}

\begin{frame}
\frametitle{Aula de reposição(?)}
    \begin{block}{}
    Conversar sobre próxima semana...
    \end{block}
\end{frame}

\end{document}