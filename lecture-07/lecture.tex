% Title page.
\title[Aula 08]{Ondas e marés}
\subtitle{Ondas influenciadas pela rotação da Terra.}
\author[Filipe Fernandes]{Filipe P. A. Fernandes}
\institute[unimonte]{Centro Universitário Monte Serrat}
\date[Outubro 2013]{25 de Outubro 2013}

\logo{\includegraphics[scale=0.15]{../common/university_logo.png}}

\begin{document}

% The title page frame.
\begin{frame}[plain]
  \titlepage
\end{frame}

\section*{Outline}
\begin{frame}
\tableofcontents
\end{frame}

\section{Ondas influenciadas pela rotação da Terra}
\subsection{Ondas de Poincaré}
\begin{frame}
\frametitle{Ondas de Poincaré}
  \begin{columns}
    \begin{column}{0.5\textwidth}
    \begin{block}{}
      Ondas sem rotação:
      \[\pd{u}{t} = -g\pd{\eta}{x}\]
      \[\pd{\eta}{t} = -h\pd{u}{x}\]
    \end{block}
    \end{column}
    \begin{column}{0.5\textwidth}
    \begin{block}{}
      \[\frac{\partial^2\eta}{\partial x^2} - \frac{1}{gh}\frac{\partial^2\eta}{\partial{t^2}} = 0\]
    \end{block}
    \end{column}
  \end{columns}
\end{frame}

\begin{frame}
\frametitle{Ondas de Poincaré}
  \begin{columns}
    \begin{column}{0.5\textwidth}
    \begin{block}{}
      Ondas com rotação em um fundo plano:
      \[\pd{u}{t} = +fv -g\pd{\eta}{x}\]
      \[\pd{v}{t} = -fu -g\pd{\eta}{y}\]
      \[\pd{\eta}{t} = -h\left(\pd{u}{x} + \pd{v}{y}\right)\]
    \end{block}
    \end{column}
    \begin{column}{0.5\textwidth}
    \begin{block}{}
      Ondas com rotação em um fundo plano:
      \[\nabla^2\hat{\eta} + \frac{(\omega^2 - f^2)}{gh} = 0\text{ onde,}\]
      \[\eta = \hat{\eta}e^{-i\omega t} \text{ e,}\]
      \[k^2 = \frac{\omega^2 - f^2}{gh}\]
    \end{block}
    \end{column}
  \end{columns}
\end{frame}

\subsection{Ondas de Kelvin}
\begin{frame}
\frametitle{Ondas de Kelvin}
{\scriptsize
  Mesma equações que a onda de Poincaré, mas assumimos $u=0$.
  \begin{columns}
    \begin{column}{0.5\textwidth}
    \begin{block}{}
      Ondas com rotação em um fundo plano:
      \[\cancelto{0}{\pd{u}{t}} = +fv -g\pd{\eta}{x}\]
      \[\pd{v}{t} = \cancelto{0}{-fu} -g\pd{\eta}{y}\]
      \[\pd{\eta}{t} = -h\left(\cancelto{0}{\pd{u}{x}} + \pd{v}{y}\right)\]
    \end{block}
    \end{column}
    \begin{column}{0.5\textwidth}
    \begin{block}{}
      Ondas com rotação em um fundo plano:
      \[\frac{\partial^2\hat{\eta}}{\partial x^2} - \frac{1}{gh}\frac{\partial^2\hat{\eta}}{\partial{t^2}} = 0\]
      \[fv = g\pd{\hat{\eta}}{x} \text{ onde,}\]
      \[\eta = \eta_o\cos(ky - \omega t)e^{-\frac{fx}{c}}\]
      \[\frac{f}{c} = \frac{f}{\sqrt{gh}} = \frac{1}{\text{Raio de deformação de Rossby}}\]
    \end{block}
    $R_d$ típico é da ordem de 2000 km no oceano aberto, 300 km para regiões
    costeiras.
    \end{column}
  \end{columns}
}
\end{frame}

\begin{frame}
  \frametitle{Representação de uma onda de Kelvin.}
  \begin{center}
    \includegraphics[scale=0.45]{../figures/rossby_radius.png}
  \end{center}
\end{frame}


\begin{frame}
  \frametitle{Representação de uma onda de Kelvin.}
  \begin{center}
    \includegraphics[scale=0.54]{../figures/kelvin.png}
  \end{center}
\end{frame}

\begin{frame}
  \frametitle{Pontos Anfidrômicos.}
  \begin{center}
    \includegraphics[scale=0.3]{../figures/amphidromes.png}
  \end{center}
\end{frame}

\subsection{Ondas de Rossby}
\begin{frame}
  \frametitle{Ondas de vorticidade -- (Rossby)}
{\scriptsize
  \begin{columns}
    \begin{column}{0.5\textwidth}
    \begin{block}{}
      \[ \pd{u}{t} = +(f_o + \beta y)v - g\pd{\eta}{x} \]
      \[ \pd{v}{t} = -(f_o + \beta y)u - g\pd{\eta}{y} \]
      \[ \pd{\eta}{t} = -h\left(\pd{u}{x} + \pd{v}{y}\right) \]
    \end{block}
    \end{column}
    \begin{column}{0.5\textwidth}
    \begin{block}{}
      \[\pd{}{t}\nabla^2\eta - \frac{f_o^2}{gh}\eta + \beta\pd{\eta}{x} = 0\text{ onde,}\]
      \[\eta = \hat{\eta}e^{i(kx + ly -\omega t)}\]
      \[\omega = \frac{-\beta k}{k^2 + l^2 + \frac{f_o^2}{gh}}\]
    \end{block}
    \end{column}
  \end{columns}
}
\vspace{0.5cm}
\pause
{\small
Ondas de Rossby tem a vorticidade como força de restauração e sempre se
propagam para oeste.

\[\Pi = \frac{\zeta + f}{H}\]
}
\end{frame}

\subsection{Teoria e mecanismos de geração de ondas de Rossby}
\begin{frame}
  \frametitle{Exemplos}
    \begin{center}
    \includegraphics[scale=0.35]{../figures/rossby_scheme.png}
  \end{center}
\end{frame}

\subsection{Importância das ondas de Rossby}
\begin{frame}
  \frametitle{Exemplos}
    \begin{center}
    \includegraphics[scale=0.45]{../figures/atmospheric_rossby.png}
  \end{center}
\end{frame}

\subsection{Observações das ondas de Rossby}
\begin{frame}
  \frametitle{Exemplos}
    \begin{center}
    \includegraphics[scale=0.45]{../figures/rossby_obs1.png}
  \end{center}
\end{frame}

\begin{frame}
  \frametitle{Exemplos}
    \begin{center}
    \includegraphics[scale=0.35]{../figures/rossby_obs2.png}
  \end{center}
\end{frame}


\begin{frame}
  \frametitle{Exemplos}
    \begin{center}
    \includegraphics[scale=0.34]{../figures/rossby_speed.png}
  \end{center}
\end{frame}


\begin{frame}
\frametitle{Dever de casa}
    Checar o portal!
\end{frame}
\section{Marés}

\end{document}
