% Title page.
\title[Aula 07]{Ondas e marés}
\subtitle{Ondas influenciadas pela rotação da Terra.}
\author[Filipe Fernandes]{Filipe P. A. Fernandes}
\institute[unimonte]{Centro Universitário Monte Serrat}
\date[Outubro 2013]{25 de Outubro 2013}

\logo{\includegraphics[scale=0.15]{../common/university_logo.png}}

\begin{document}

% The title page frame.
\begin{frame}[plain]
  \titlepage
\end{frame}

\section*{Outline}
\begin{frame}
\tableofcontents
\end{frame}

\section{Ondas influenciadas pela rotação da Terra}
\subsection{Modelo de Águas rasas}
\begin{frame}
  \frametitle{Modelo de Águas rasas}
  \begin{center}
    \includegraphics[scale=0.7]{../figures/aguas_rasas.pdf}
  \end{center}
\end{frame}

\subsection{Ondas de Poincaré}
\begin{frame}
\frametitle{Ondas longas sem rotação}
  \begin{columns}
    \begin{column}{0.5\textwidth}
    \begin{block}{}
      Ondas sem rotação:
      \[
        \pd{u}{t} = -g\pd{\eta}{x}
      \]
      \[
        \pd{\eta}{t} = -h\pd{u}{x}
      \]
    \end{block}
    \end{column}
    \begin{column}{0.5\textwidth}
    \begin{block}{}
      \[
        \frac{\partial^2\eta}{\partial x^2} - \frac{1}{gh}\frac{\partial^2\eta}{\partial{t^2}} = 0
      \]
    \end{block}
    \end{column}
  \end{columns}
\end{frame}

\begin{frame}
\frametitle{Ondas de Poincaré}
  \begin{columns}
    \begin{column}{0.5\textwidth}
    \begin{block}{}
      Ondas com rotação em um fundo plano:
      \[
        \pd{u}{t} - fv = -g\pd{\eta}{x}
      \]
      \[
        \pd{v}{t} + fu= -g\pd{\eta}{y}
      \]
      \[
        \pd{\eta}{t} = -h\left(\pd{u}{x} + \pd{v}{y}\right)
      \]
    \end{block}
    \end{column}
    \begin{column}{0.5\textwidth}
    \begin{block}{}
      Ondas com rotação em um fundo plano:
      \[
        \nabla^2\eta + \frac{(\omega^2 - f^2)}{gh} = 0\text{ onde,}
      \]
      \[
        \eta = \hat{\eta}e^{-i\omega t} \text{ e,}
      \]
      \[
        \omega^2 = f^2 + {gh}k^2
       \]
    \end{block}
    \end{column}
  \end{columns}
\end{frame}


\begin{frame}
  \frametitle{Resumo de ondas de Poincaré}
\small{
  \begin{itemize}[<+-| alert@+>]
    \item Se progaga no oceano aberto em qualquer direção;
    \item É uma onda dispersiva.
%     \item For a plane wave crests and troughs have uniform height.
%     \item Short waves act like non-rotating shallow water gravity waves
    \item Os vetores do fluxo giram de forma anti-ciclônica
    \item O balanço de forças na direção de propação da ondas incluí o
          gradiente de pressão e aceleração do fluxo, mas também a força de
          Coriolis devido as velocidades do fluxo ao longo das cristas;
    \item O balanço de força na direção paralela as cristas e cavas é a
          aceleração de Coriolis devido a direção do fluxo.
  \end{itemize}
}
\end{frame}

\subsection{Ondas de Kelvin}
\begin{frame}
\frametitle{Ondas de Kelvin}
{\scriptsize
  Mesma equações que a onda de Poincaré, mas assumimos $u=0$.
  \begin{columns}
    \begin{column}{0.5\textwidth}
    \begin{block}{}
      Ondas com rotação em um fundo plano:
      \[
          \cancelto{0}{\pd{u}{t}} = +fv -g\pd{\eta}{x}
       \]
      \[
          \pd{v}{t} = \cancelto{0}{-fu} -g\pd{\eta}{y}
       \]
      \[
          \pd{\eta}{t} = -h\left(\cancelto{0}{\pd{u}{x}} + \pd{v}{y}\right)
       \]
    \end{block}
    \end{column}
    \begin{column}{0.5\textwidth}
    \begin{block}{}
      Ondas com rotação em um fundo plano:
      \[
          \frac{\partial^2\eta}{\partial y^2} = \frac{1}{gh}\frac{\partial^2\eta}{\partial{t^2}}
       \]
      \[
          fv = g\pd{\eta}{x} \text{ onde,}
       \]
      \[
          \eta = \eta_o\cos(ky - \omega t)e^{-\frac{fx}{c}}
       \]
      \[
          \frac{f}{c} = \frac{f}{\sqrt{gh}} = \frac{1}{\text{Raio de deformação de Rossby}}
       \]
      \[
          \omega = -l\sqrt{gh}
      \]

    \end{block}
    $R_d$ típico é da ordem de 2000 km no oceano aberto, 300 km para regiões
    costeiras.
    \end{column}
  \end{columns}
}
\end{frame}

\begin{frame}
  \frametitle{Representação de uma onda de Kelvin}
  \begin{center}
    \includegraphics[scale=0.45]{../figures/rossby_radius.png}
  \end{center}
\end{frame}


\begin{frame}
  \frametitle{Representação de uma onda de Kelvin}
  \begin{center}
    \includegraphics[scale=0.65]{../figures/kelvin.png}
  \end{center}
\end{frame}

\begin{frame}
  \frametitle{Pontos Anfidrômicos.}
  \begin{center}
    \includegraphics[scale=0.3]{../figures/amphidromes.png}
  \end{center}
\end{frame}


\begin{frame}
  \frametitle{Resumo de ondas de Kelvin}
\small{
  \begin{itemize}[<+-| alert@+>]
  \item Propaga ao longo da costa de uma forma a manter a costa
        a sua esquerda (direta) no Hemisfério Sul (Norte);
  \item A velocidade de fase e grupo é a mesma das ondas longas
        sem a influência da rotação da Terra (C = Cg = $\sqrt{gh}$);
  \item As crista e cavas da onda são paralelas à costa;
  \item As crista decaem exponencialmente para longe da costa na escala do Raio
        de deformação de Rossby;
  \item O balanço de forças perpendicular à costa é Geostrófico;
  \item O balanço de forças ao longo da costa é de uma onda de gravidade típica:
      Onda variações na pressão guiam a aceleração do fluxo.
  \end{itemize}
}
\end{frame}


\subsection{Ondas de Rossby}
\begin{frame}
  \frametitle{Ondas de Rossby (ou de vorticidade)}
{\scriptsize
  \begin{columns}
    \begin{column}{0.5\textwidth}
    \begin{block}{}
      \[
        \pd{u}{t} - (f_o + \beta y)v = - g\pd{\eta}{x}
      \]
      \[
        \pd{v}{t} + (f_o + \beta y)u = - g\pd{\eta}{y}
      \]
      \[
        \pd{\eta}{t} + h\left(\pd{u}{x} + \pd{v}{y}\right)= 0
      \]
    \end{block}
    \end{column}
    \begin{column}{0.5\textwidth}
    \begin{block}{}
      \[
        \pd{}{t}\nabla^2\eta - \frac{f_o^2}{gh}\eta + \beta\pd{\eta}{x} = 0\text{ onde,}
      \]
      \[
        \eta = \hat{\eta}e^{i(kx + ly -\omega t)}
      \]
      \[
        \omega = \frac{-\beta k}{k^2 + l^2 + \frac{1}{R^2}}
      \]

      $R^2 = \frac{gh}{f_o^2}$
    \end{block}
    \end{column}
  \end{columns}
}
\vspace{0.5cm}
\pause
{\small Ondas de Rossby tem a vorticidade como força de restauração e sempre se
        propagam para oeste.}
\[
  \Pi = \frac{\zeta + f}{H}
\]

\end{frame}

\subsection{Teoria e mecanismos de geração de ondas de Rossby}
\begin{frame}
  \frametitle{Exemplos}
    \begin{center}
    \includegraphics[scale=0.35]{../figures/rossby_scheme.png}
  \end{center}
\end{frame}

\subsection{Importância das ondas de Rossby}
\begin{frame}
  \frametitle{Exemplos}
    \begin{center}
    \includegraphics[scale=0.45]{../figures/atmospheric_rossby.png}
  \end{center}
\end{frame}

\subsection{Observações das ondas de Rossby}
\begin{frame}
  \frametitle{Exemplos}
    \begin{center}
    \includegraphics[scale=0.45]{../figures/rossby_obs1.png}
  \end{center}
\end{frame}

\begin{frame}
  \frametitle{Exemplos}
    \begin{center}
    \includegraphics[scale=0.35]{../figures/rossby_obs2.png}
  \end{center}
\end{frame}


\begin{frame}
  \frametitle{Exemplos}
    \begin{center}
    \includegraphics[scale=0.34]{../figures/rossby_speed.png}
  \end{center}
\end{frame}

\begin{frame}
  \frametitle{Propriedades das Ondas de Rossby}
\small{
  \begin{itemize}[<+-| alert@+>]
  \item $k < 0$  significa que Ondas de Rossby têm velocidades de fase
        para Oeste;
  \item A velocidade de grupo pode ser para Oeste ou Leste;
  \item As ondas longas tem velocidade de grupo para Oeste;
  \item As ondas curtas tem velocidade de grupo para Leste;
  \item Para certos valores de $R$ e $\beta$ tem um máximo valor de $\omega$.
        Se você mover para Norte esse valor decresce, logo, ondas podem ficar
        pressas no Equador.
  \item Essas ondas são sub-inerciais $\omega < f$.
  \end{itemize}
}
\end{frame}

\begin{frame}
\frametitle{Dever de casa}
    Escrever as equações para as ondas de Rossby, Kelvin e Poincaré lado-a-lado e criar uma
    tabela sobre as suas diferenças e semelhanças.
\end{frame}

\end{document}
