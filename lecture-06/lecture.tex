% Title page.
\title[Aula 06]{Ondas e marés}
\subtitle{Análise de Fourier}
\author[Filipe Fernandes]{Filipe P. A. Fernandes}
\institute[unimonte]{Centro Universitário Monte Serrat}
\date[Agosto 2013]{18 de Outubro 2013}

\logo{\includegraphics[scale=0.15]{../common/university_logo.png}}

\begin{document}

% The title page frame.
\begin{frame}[plain]
  \titlepage
\end{frame}

\section*{Outline}
\begin{frame}
\tableofcontents
\end{frame}

\section{Fourier}
\begin{frame}
\frametitle{Joseph Fourier (1907)}
  \begin{itemize}[<+-| alert@+>]
    \item Medidas: Sinais periódicas + Ruído aleatório, ou
    \item Medidas: Parte determinística + parte estocástica.
    \item Objetivo:
      \begin{enumerate}[<+-| alert@+>]
        \item Separar a parte periódica do ruído,
        \item Quantificar a variância associada a cada componente da parte
              periódica.
      \end{enumerate}
    \item Requisitos:
      \begin{enumerate}[<+-| alert@+>]
        \item Série temporal contínua;
        \item Amplitudes e fases variam de forma suave;
        \item Sinais estacionários.
      \end{enumerate}
  \end{itemize}
\end{frame}


\begin{frame}
\frametitle{Fourier}
    \begin{block}{}
      Premissa: $y(t) = \overline{y(t)} + \sum_i\left[ A_i\cos(\omega_it) + B_i\sin(\omega_it) \right]$
    \end{block}
    \pause
    \begin{block}{}
      Queremos determinar: $A_i, B_i, \omega_i = \frac{2\pi i}{T}$, $i=1,2,3...$
    \end{block}
\end{frame}


\begin{frame}
\frametitle{Transformada contínua}
\footnotesize{
  \begin{itemize}[<+-| alert@+>]
    \item Partindo de $y(t) = \sum_i\left[ A_i\cos(\omega_it) + B_i\sin(\omega_it) \right]$
    \item Multiplicando por $\cos(\omega_it)$ e integrando de 0 a $T$:\\
    $$A_i = \frac{2}{T}\int^T_0y(t)\cos(\omega_it)dt, i=0,1,2...$$
    \item Multiplicando por $\sin(\omega_it)$ e integrando de 0 a $T$:\\
    $$B_i = \frac{2}{T}\int^T_0y(t)\sin(\omega_it)dt, i=1,2,3...$$
    \item A transformada contínua é:
    $$y(t) \frac{1}{2}A_o + \sum^{\infty}_i=1\left[A_i\cos(\omega_it) + B_i\sin(\omega_it)\right], i=1,2,3...$$
    \end{itemize}
    }
\end{frame}


\begin{frame}
\frametitle{Em termos de amplitude e fase}
\footnotesize{
  \begin{itemize}[<+-| alert@+>]
    \item A soma de senos e cossenos dada por:
    $$y(t) \frac{1}{2}A_o + \sum^{\infty}_i=1\left[A_i\cos(\omega_it) + B_i\sin(\omega_it)\right], i=1,2,3...$$
    \item É equivalente a:\\
    $$y(t)=\frac{1}{2}C_o + \sum^{\infty}_i=1\left[C_i\cos(\omega_it-\phi_i)\right], i=1,2,3...$$
    \item Amplitude: $C_i = \sqrt{A^2_i + B^2_i}, i=0,1,2...$
    \item Fase: $\phi_i = \arctan\left({\frac{B_i}{A_i}}\right), i=1,2,3...$
    \end{itemize}
    }
\end{frame}


\begin{frame}
\frametitle{Dever de casa.}
    \begin{block}{}
    Checar no sistema.
    \end{block}
\end{frame}

\end{document}
