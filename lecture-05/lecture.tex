
% Title page.
\title[Aula 05]{Ondas e Marés}
\subtitle{Fenômenos associados à propagação de ondas.}
\author[Filipe Fernandes]{Filipe P. A. Fernandes}
\institute[unimonte]{Centro Universitário Monte Serrat}
\date[Setembro 2013]{27 de Setembro 2013}

\logo{\includegraphics[scale=0.15]{../common/university_logo.png}}

\begin{document}

% The title page frame.
\begin{frame}[plain]
  \titlepage
  \begin{center}
    \includegraphics[scale=0.5]{../common/by-nc-sa.pdf}
  \end{center}
\end{frame}

\section*{Outline}
\begin{frame}
\tableofcontents
\end{frame}

\section{Aula 05 -- Fenômenos associados à propagação de ondas.}
\begin{frame}
\frametitle{Conteúdo até agora}
{\scriptsize
  \begin{itemize}[<+-| alert@+>]
    \item[1] Introdução: O que são ondas e como são geradas?
    \item[1.1] Definições
    \item[1.1.1] Elevação
    \item[1.1.2] Altura de onda
    \item[1.1.3] comprimento de onda
    \item[1.1.4] Número de onda
    \item[1.1.5] Período e frequência
    \item[1.1.6] Frequência angular
    \item[1.1.7] Velocidade de fase
    \item[1.1.8] Velocidade de grupo
    \item[2] Ondas na superfície do mar
    \item[2.1] Tipos de ondas
    \item[2.2] Medição de ondas de superfície
    \item[3] Ondas de gravidade
  \end{itemize}
  }
\end{frame}


\begin{frame}
\frametitle{Conteúdo de hoje}
{\footnotesize
  \begin{itemize}[<+-| alert@+>]
    \item[4] Fenômenos associados à propagação das ondas na superfície do mar
    \item[4.1] Dispersão
    \item[4.2] Interferência
    \item[4.3] Velocidade de grupo de ondas de gravidade em águas profundas
    \item[4.4] Reflexão das ondas
    \item[4.5] Refração das ondas
    \item[4.6] Difração das ondas
    \item[4.7] Relação das ondas com o vento
    \item[4.8] Crescimento das ondas geradas pelo vento
    \item[4.9] Quebra de ondas
  \end{itemize}
  }
\end{frame}


\subsection{Revisão -- Ondas lineares}
\begin{frame}
\frametitle{Revisão -- Ondas lineares}
  \begin{itemize}[<+-| alert@+>]
    \item $\frac{u}{c} << 1$, isso significa que as pertubações são muito
          fracas em relação ao fluxo médio (\bf{aproximação das pertubações});
    \item O sistema é não-dissipativo.  As ondas não são amortecidas quando a
          duração da onda é muito maior que o período ({\bf sem atrito});
    \item O sistema não é forçado ({\bf sempre longe da geração das ondas}).
    \item O período da onda é muito menor que a rotação da Terra ({\bf podemos desprezar o efeito de Coriolis}).
  \end{itemize}

\end{frame}


\subsection{Dispersão}
\begin{frame}
\frametitle{Dispersão}
    \begin{block}{}
    O espectro de frequências das ondas no oceano se estendem desde movimentos
    ondulatórios capilares (1 s), até ondas planetárias de baixa frequência
    com períodos de alguns anos.
    \end{block}
\end{frame}


\begin{frame}
\frametitle{Frequência}
  \begin{center}
    \includegraphics[scale=0.2]{../figures/espectros.png}
  \end{center}

  {\tiny Espectro de energia da variabilidade oceânica, mostrando os diferentes
         tipos de ondas que ocorrem no oceano.}

  {\tiny Fonte: LeBlond e Mysak (1978).}
\end{frame}



\begin{frame}
\frametitle{Dispersão 01}
\begin{center}
  \includegraphics[scale=0.35]{../figures/dispersion.png}
\end{center}
\end{frame}


\begin{frame}
\frametitle{Dispersão 02}
\begin{center}
  \includegraphics[scale=0.9]{../figures/dispersion_01.png}
\end{center}
\end{frame}


\begin{frame}
\frametitle{Dispersão 03}
\begin{center}
  \includegraphics[scale=0.2]{./figures/2006-01-14_Surface_waves.jpg}
% http://upload.wikimedia.org/wikipedia/commons/4/43/2006-01-14_Surface_waves.jpg
\end{center}
\end{frame}


\begin{frame}
\frametitle{Implicações da dispersão}
  {\small
  \begin{itemize}[<+-| alert@+>]
    \item Tempestade: As ondas mais longas chegaram primeiro;
    \item O período vai mudar período {\bf NUNCA} muda;
    \item Em um grupo de ondas com $\approx \lambda$ a energia viajará a
          $\frac{1}{2}$ da velocidade das ondas individuais;
    \item Ondas que viajam de tempestades ``longes'' são chamadas de ``swell'';
    \item Ondas geradas no loca são chamdas de ``wind-sea'';
    \item Separar essas duas das medições e complicado.
  \end{itemize}
  }
\end{frame}


\subsection{Interferência 01}
\begin{frame}
\frametitle{Interferência}
\begin{center}
  \includegraphics[scale=0.35]{../figures/interference.png}
\end{center}
\end{frame}


\begin{frame}
\frametitle{Interferência 02}
\begin{center}
  \includegraphics[scale=0.3]{./figures/800px-ile_de_re.jpg}
% http://en.wikipedia.org/wiki/File:Ile_de_r%C3%A9.JPG
\end{center}
\end{frame}


\begin{frame}
\frametitle{Interferência 03}
\begin{center}
  \includegraphics[scale=0.6]{./figures/interference_moher.pdf}
\end{center}
\end{frame}


\begin{frame}
  \frametitle{Demostração com duas ondas}
  \begin{block}{}
    Demonstrar ``two\_wave.py''
  \end{block}
\end{frame}

\begin{frame}
\frametitle{Ondas se aproximando da costas}
\begin{center}
  \includegraphics[scale=0.6]{../figures/coastal_approach.png}
\end{center}
\end{frame}


\subsection{Refração das ondas}
\begin{frame}
\frametitle{Lei de Snell}
  \begin{center}
    \includegraphics[scale=0.4]{../figures/refraction.png}
  \end{center}
\end{frame}


\begin{frame}
\frametitle{Ondas em um ângulo oblíquo com a costa}
  \begin{center}
    \includegraphics[scale=0.8]{../figures/refraction_01.png}
  \end{center}
\end{frame}


\begin{frame}
\frametitle{Refração 01}
  \begin{center}
    \includegraphics[scale=0.65]{../figures/refraction_02.png}
  \end{center}
\end{frame}


\begin{frame}
\frametitle{Refração 02}
  \begin{center}
    \includegraphics[scale=0.6]{./figures/refraction_moher.pdf}
  \end{center}
\end{frame}


\subsection{Difração das ondas}
\begin{frame}
\frametitle{Difração 01}
  {\scriptsize É a tendência das ondas de se ``dobrarem'' ao redor de
               obstruções.}
  \begin{center}
    \includegraphics[scale=0.6]{../figures/difraction.png}
  \end{center}
\end{frame}


\begin{frame}
\frametitle{Difração 02}
  \begin{center}
    \includegraphics[scale=0.6]{../figures/difraction_01.png}
  \end{center}
\end{frame}


\begin{frame}
\frametitle{Difração 03}
  \begin{center}
    \includegraphics[scale=0.7]{../figures/difraction_02.png}
  \end{center}
\end{frame}


\begin{frame}
\frametitle{Difração 04}
\begin{center}
  \includegraphics[scale=1.4]{../figures/difraction_photo.png}
\end{center}
\end{frame}


\begin{frame}
\frametitle{Difração 05}
\begin{center}
  \includegraphics[scale=1.6]{../figures/gibraltar.png}
\end{center}
\end{frame}


\begin{frame}
\frametitle{Difração 06}
\begin{center}
  \includegraphics[scale=0.6]{./figures/difraction_moher.pdf}
\end{center}
\end{frame}


\begin{frame}
\frametitle{Difração e Refração}
\begin{center}
  \includegraphics[scale=0.6]{./figures/difraction_refraction_santos.pdf}
\end{center}
\end{frame}


\subsection{Relação das ondas com o vento}
\begin{frame}
\frametitle{Ondas geradas pelo vento}
  \begin{block}{}
    Quando o vento sopra sobre a superfície do oceano, ondas de superfície são
    geradas por transferência de momento do vento, indo do ar para a água.
    Uma onda real no oceano pode ser composta por diversas componentes de ondas
    com diferentes períodos e alturas.

    Dessa forma, a medida de ondas no oceano não é uma tarefa fácil.
  \end{block}
\end{frame}

\begin{frame}
\frametitle{Ondas geradas pelo vento}
  Para entender como as ondas se desenvolvem, devemos considerar quatro
  fatores:
  \begin{itemize}[<+-| alert@+>]
    \item[1.] A velocidade do vento;
    \item[2.] Pista do vento ou a distância em que o vento sopra.  Em inglês,
              {\it fetch};
    \item[3.] Duração do vento;
    \item[4.] Profundidade da água.
  \end{itemize}
\end{frame}


\begin{frame}
\frametitle{Ondas geradas pelo vento}
  \begin{center}
    \includegraphics[scale=0.25]{../figures/wind_speed_wave.png}
  \end{center}
  \tiny{Espectro de energia da onda para um campo plenamente desenvolvido\\
        no mar sob a ação de diferentes magnitudes da velocidade do vento\\
        De acordo com Pierson e Moskovitz (1964).}
\end{frame}


\subsection{Crescimento das ondas geradas pelo vento}
\begin{frame}
\frametitle{Processo de geração}
  {\small O processo de geração de ondas pelo vento passa por pelo menos três}
          estágios:
  {\scriptsize
  \begin{itemize}[<+-| alert@+>]
    \item[1.] A turbulência no vento gera flutuações randômicas na pressão da
              superfície do mar que por conseguinte geram pequenas ondas com
              comprimentos de onda em torno de poucos cm;
    \item[2.] Em seguida, o vento age sobre essas ondas pequenas e por
              processos de interação fazem com que elas cresçam. Isso ocorre
              porque a ação do vento sobre uma onda causa diferenças de pressão
              ao longo do perfil da onda fazendo com que elas cresçam. Esse
              processo é um processo instável pois quanto maior a onda, maior a diferença de pressão e mais rapidamente a onda cresce. Essa instabilidade faz com que a onda cresça exponencialmente.
    \item[3.] Finalmente, as ondas começam a interagir entre elas. A interação
              transfere a energia das ondas mais curtas para ondas com
              frequências um pouco menores que as do pico do espectro.
              Eventualmente, isso faz com que as ondas se propaguem até mais
              rapidamente que o próprio vento que as gerou.
  \end{itemize}
  }
\end{frame}


\subsection{Altura significante de onda}
\begin{frame}
\frametitle{Altura significante de onda}
  {\small
  \begin{block}{}
    Num oceano real, a altura de uma onda é determinada pela combinação de
    várias ondas, com diferentes frequências e amplitudes que se movem
    relativamente umas em relação às outras, podendo estar em fase ou não.

    Teoricamente, se as alturas e as frequências de cada onda que contribui no
    movimento forem conhecidas, então é possível prever as alturas e as
    frequências das ondas reais com bastante precisão. Na prática, isso é
    raramente possível.
  \end{block}
  }
\end{frame}


\begin{frame}
\frametitle{Altura significante de onda}
  {\small
  \begin{block}{}
  $H_1/3$, é a altura média das ondas da terça parte com as maiores ondas que
  ocorrem numa determinada região para um determinado período.  Podemos
  encontrar um máximo na altura da onda nesses registros, $H_{max}$.

  A previsão de $H_{max}$ para um dado período é uma informação muito
  importante para o planejamento de estruturas como barreiras, cais e
  plataformas de perfuração.
  \end{block}
  }
\end{frame}


\begin{frame}
\frametitle{Altura significante de onda}
  \begin{center}
    \includegraphics[scale=0.35]{../figures/altura_de_ondas.png}
  \end{center}
\end{frame}


\subsection{Quebra de ondas}
\begin{frame}
\frametitle{Quebra de ondas}
  \begin{block}{}
    O que acontece quando nossa ondas não obedecem mais a restrição de
    $\frac{H}{L} << 1$?
  \end{block}
\end{frame}

\begin{frame}
\frametitle{Quebra de ondas}
  \begin{center}
    \includegraphics[scale=0.6]{../figures/break_01.png}
  \end{center}
\end{frame}


\begin{frame}
\frametitle{Spilling}
  \begin{center}
    \includegraphics[scale=0.6]{../figures/break_02.png}
  \end{center}
\end{frame}


\begin{frame}
\frametitle{Plunging}
  \begin{center}
    \includegraphics[scale=0.6]{../figures/break_03.png}
  \end{center}
\end{frame}


\begin{frame}
\frametitle{Surging}
  \begin{center}
    \includegraphics[scale=0.6]{../figures/break_04.png}
  \end{center}
\end{frame}


\begin{frame}
\frametitle{Dever de casa}
    \begin{block}{}
      Relação de dispersão de ondas internas.
    \end{block}
\end{frame}

\end{document}
